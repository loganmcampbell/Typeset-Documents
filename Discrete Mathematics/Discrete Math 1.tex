\documentclass[10pt,letterpaper]{article}
\usepackage{fullpage}
\usepackage[top=2cm, bottom=4.5cm, left=2.5cm, right=2.5cm]{geometry}
\usepackage{amsmath,amsthm,amsfonts,amssymb,amscd}
\usepackage{lastpage}
\usepackage{enumerate}
\usepackage{fancyhdr}
\usepackage{mathrsfs}
\usepackage{mathtools}
\usepackage{xcolor}
\usepackage{graphicx}
\usepackage{listings}
\usepackage{hyperref}

\hypersetup{%
  colorlinks=true,
  linkcolor=blue,
  linkbordercolor={0 0 1}
}
\newcommand{\floor}[1]{\left\lfloor #1 \right\rfloor}
\newcommand{\ceil}[1]{\left\lceil #1 \right\rceil}


\lstdefinestyle{Python}{
    language        = Python,
    frame           = lines, 
    basicstyle      = \footnotesize,
    keywordstyle    = \color{blue},
    stringstyle     = \color{green},
    commentstyle    = \color{red}\ttfamily
}

\setlength{\parindent}{0.0in}
\setlength{\parskip}{0.05in}

% Edit these as appropriate
\newcommand\course{CSCE 4133}
\newcommand\hwnumber{}                  % <-- homework number
\newcommand\NetIDa{Logan Campbell}           % <-- NetID of person #1
\newcommand\NetIDb{010 - 641 - 227}           % <-- NetID of person #2 (Comment this line out for problem sets)

\pagestyle{fancyplain}
\headheight 25pt
\lhead{LOGAN CAMPBELL}
\chead{}
\lfoot{}
\cfoot{}
\rfoot{\small\thepage}
\headsep 1.0em
\begin{document}

{

\begin{enumerate}
    \item[\#1.] Complete the Truth tables for $\bar{p} \rightarrow{}q, \;\; p \vee q \;$ and $(p \wedge q) \rightarrow{} \bar{r}$
    
    
\begin{table}[h!]
    \label{tab:table1}
    \begin{tabular}{l|c|r} % <-- Alignments: 1st column left, 2nd middle and 3rd right, with vertical lines in between
      %\textbf{Value 1} & \textbf{Value 2} & \textbf{Value 3}\\
      $p$ & $q$ & $\bar{p} \rightarrow{q}$ \\
      \hline
      T & T & \\
      T & F & \\
      T & T & \\
      T & F & \\
    \end{tabular}
\end{table}

\begin{table}[h!]
    \label{tab:table2}
    \begin{tabular}{l|c|r} % <-- Alignments: 1st column left, 2nd middle and 3rd right, with vertical lines in between
      %\textbf{Value 1} & \textbf{Value 2} & \textbf{Value 3}\\
      $p$ & $q$ & $p \vee q$ \\
      \hline
      T & T & \\
      T & F & \\
      T & T & \\
      T & F & \\
    \end{tabular}
\end{table}

\begin{table}[h!]
    \label{tab:table3}
    \begin{tabular}{l|c|c|r} % <-- Alignments: 1st column left, 2nd middle and 3rd right, with vertical lines in between
      %\textbf{Value 1} & \textbf{Value 2} & \textbf{Value 3}\\
      $p$ & $q$ & $r$ & $(p \wedge q) \rightarrow{\bar{r}}$ \\
      \hline
      T & T & T\\
      T & T & F\\
      T & F & T\\
      T & F & F\\
      F & T & T\\
      F & T & F\\
      F & F & T\\
      F & F & F\\
    \end{tabular}
\end{table}

\begin{flushleft}
Are $\bar{p} \rightarrow{q} \text{ and } p \wedge q \text{ logically equivalent?}$
\end{flushleft}

\vspace{4em}
\item[\#2.] Show that if $3x^{2} - 6x > 0 \text{ , then } x \neq 2. $ Prove by a contrapositive :
\vspace{4em}
\item[\#3.] Write each of the following in mathematical notation and determine whether it is \underline{true} or \underline{false}.
\vspace{1em}

a. for every $x$ $(\text{in }\mathbb{R})$ there exists $y$ $(\text{in }\mathbb{R})$ such that $x \cdot y \geq 0$
\vspace{4em}

b. there exists $x$ $(\text{in }\mathbb{R})$ and $y$ $(\text{in }\mathbb{R})$ such that $x^{2} + y^{2} < 2xy$
\vspace{4em}



\end{enumerate}
}





\newpage{}
{

\begin{enumerate}
	\item[\#4.] Using DeMorgan's Laws for logic, write \textbf{negation} for the following statement in English:
	\begin{center}
	    $\forall \: x \: \exists \: y, x^{2} \cdot y > 0 $
	\end{center}
\vspace{4em}

\item[\#5.] Write the \textbf{converse} of \textbf{A} and the \textbf{contrapostive} of \textbf{B}.

\textbf{A.} If you can run and jump, then you can walk.
\vspace{3em}

\textbf{B.} If $x^{2} - 1 = 0$, then $x = -1 \text{ or } x = 1.$
\vspace{3em}

\item[\#6.] Use the euclidean algorithm to determine \textit{gcd}(456,221). If 456 and 221 are \textit{relatively prime}, then work back through the euclidean algorithm to find integers \textit{k} and \textit{l} . such that \textit{k} 456 +\textit{l}221 = 1.
\vspace{4em}

\item[\#7.] Define $ f : \mathbb{R} \rightarrow{\mathbb{R}}$ by $f(x) = \frac{1}{1+x^{2}}$. Let \textit{A} = \{3,4\} and let \textit{B} = \{1,$\frac{1}{2}$,$\frac{1}{5}$\}. Determine the image of \textit{A} under $f$ ($f(\textit{A})$) and the pre-image of \textit{B} under $f$ ($f^{-1}(\textit{B})$).
\vspace{2em}

A. $f(\textit{A})$ =
\vspace{1em}

B. $f^{-1}(\textit{B})$ = 
\vspace{4em}

\item[\#8.] Prove by \textbf{induction} that : $$\sum_{k =1}^{n} k^{3} = \Big[ \frac{n(n+1)^{2}}{2} \Big] ^{2}$$



\end{enumerate}
}

\newpage{}
{

\begin{enumerate}
	\item[\#9.] Classify the following relations on $X = \{ 1, 2, 3 \}$ by placing next to each the appropriate letters:
	\begin{center}(r) reflexive  (s) symmetric  (a) anti-symmetric  (t) transitive  (e) equivalence relation  (p) partial order  (o) total order
	\end{center}
\vspace{.5em}

a. $R_{1} = \{ (1,1) , (2,2) , (3,3) \}$

b. $R_{2} = \{ (1,2) , (2,3) , (1,3), (3,1) \}$

c. $R_{3} = \{ (1,1) , (2,2) , (3,3), (1,2), (2,3), (1,3)\}$

\item[\#10.] With $R_{2}$ and $R_{3}$ as given above (in problem 9), evaluate $R_{3} \circ R_{2}$.
\vspace{4em}

\item[\#11.] Prove by \textbf{induction} that $10\cdot11^{n} -3\cdot4^{n}$ is divisble by 7, for $n = 1, 2, 3, \dots$
\vspace{17em}

\item[\#12.] Define $R$ on $X \coloneqq \{1,2,3,\dots, 15\}$ by $(x,x^{\prime}) \in R$ if and only if 4 divides $(x^{\prime} - x)$. Prove that $R$ is an equivalence relation on $X$ and then determine the partition of $X$ generated by $R$.
\end{enumerate}



}
\end{document}

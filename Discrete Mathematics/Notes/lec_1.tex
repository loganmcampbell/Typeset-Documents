\documentclass[10pt,letterpaper]{article}
\usepackage{fullpage}
\usepackage[top=2cm, bottom=4.5cm, left=2.5cm, right=2.5cm]{geometry}
\usepackage{amsmath,amsthm,amsfonts,amssymb,amscd}
\usepackage{lastpage}
\usepackage{enumerate}
\usepackage{fancyhdr}
\usepackage{mathrsfs}
\usepackage{mathtools}
\usepackage{xcolor}
\usepackage{graphicx}
\usepackage{listings}
\usepackage{hyperref}
\usepackage{pifont}
\newcommand{\xmark}{\ding{55}}%
\pagestyle{fancyplain}
\headheight 25pt
\lhead{LOGAN CAMPBELL \\ Discrete Mathematics : Lecture Notes (8/26/2019)}
\chead{}
\lfoot{}
\cfoot{}
\rfoot{\small\thepage}
\headsep 1.0em
\begin{document}

{

\begin{enumerate}
    \item[]
    \begin{center}
            LOGIC
    \end{center}
    \vspace{1.0em}
    
    DEFINE: a \underline{proposition} is a true or false statement but  not both.
    
    - true proposition := \quad these letters are black \hspace{1em} which is true \checkmark
    
    - false proposition := \quad I'm not using the English language right now \hspace{1em} which is false \xmark.
    
    - none proposition := \quad a question, command, or an opinion. 
    
    \vspace{1em}

    \textit{Mathematical Propositions}
    $$1 + 3 = 4 \quad \checkmark $$
    $$1 + 4 = 10 \quad \text{\xmark} $$
    
    \begin{center}

    \begin{table}[h!]
    \label{tab:table1}
    
    \begin{tabular}{c|c} % <-- Alignments: 1st column left, 2nd middle and 3rd right, with vertical lines in between
      %\textbf{Value 1} & \textbf{Value 2} & \textbf{Value 3}\\
      SYMBOL & NEGATION     \\
      \hline
      $<$   &   $\geq$      \\
      $>$   &   $\leq$      \\
      $=$   &   $\neq$      \\
      \hline
      $\leq$ &  $>$         \\
      $\geq$ &  $<$         \\
      $\neq$ &  $=$         \\
        \end{tabular}
    \end{table}
\end{center}
\vspace{1em}

DEFINE : $p$ and $q$ propositions, conjunctions \& disjunctions and negations

 - p ``and'' q = $\wedge$
 
 - p ``or'' q  = $\vee$
 
 - negation = $\bar{p}$ or $\bar{q}$ \quad the opposite logic of p and q
 
 \vspace{1em}
 
 \begin{table}[h!]
    \label{tab:table2}
    \begin{tabular}{c|c|c|c} % <-- Alignments: 1st column left, 2nd middle and 3rd right, with vertical lines in between
      %\textbf{Value 1} & \textbf{Value 2} & \textbf{Value 3}\\
      $p$ & $q$ & ${p} \wedge {q}$  &  ${p} \vee {q}$\\
      \hline
      T & T &  T & T\\
      T & F &  F & T\\
      F & T &  F & T\\
      F & F &  F & F\\
    \end{tabular}
\end{table}
 
    
\end{enumerate}
}


\end{document}

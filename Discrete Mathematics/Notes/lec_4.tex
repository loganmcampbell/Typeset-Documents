\documentclass[10pt,letterpaper]{article}

\usepackage{fullpage}
\usepackage[top=2cm, bottom=4.5cm, left=2.5cm, right=2.5cm]{geometry}
\usepackage{amsmath,amsthm,amsfonts,amssymb,amscd}
\usepackage{lastpage}
\usepackage{enumerate}
\usepackage{empheq}
\usepackage{fancyhdr}
\usepackage{mathrsfs}
\usepackage{mathtools}
\usepackage{xcolor}
\usepackage{graphicx}
\usepackage{caption}
\usepackage{listings}
\usepackage{hyperref}
\usepackage{pifont}
\newcommand{\xmark}{\ding{55}}%
\pagestyle{fancyplain}
\headheight 25pt
\lhead{LOGAN CAMPBELL \\ Discrete Mathematics : Lecture Notes (9/4/2019)}
\chead{}
\lfoot{}
\cfoot{}
\rfoot{\small\thepage}
\headsep 1.0em
\begin{document}

{

\begin{enumerate}
    \item[]
    \begin{center}
                QUANTIFIERS
    \end{center}
    \vspace{1.0em}
    \item[] Definition: Let $S$ be a set and for each $x$ in $S$ let $p(x)$ be a proposition involving $x$. The $p$ is a called a propositional function on $S$.
    
    Example : \quad For every real number ($\mathbb{R}$) $x$, \quad $x^{2} \geq 0.$
    
    \qquad $S = \mathbb{R} \leftarrow \text{the real \#'s}$
    
    \qquad $p(x) =  x^{2} \geq 0 \leftarrow \text{ the propositional function }$
    
    another example: $S = \{ \text{ALL GEESE} \}$ \quad $x \in S$ \quad $P(x): x \text{ can fly}$
    
    \vspace{1em}
    
    \begin{center}
        \textit{FOR ALL AND THERE EXISTS}
    \end{center}
    \item[] Definition: The statement ``There exists'' $x$ in $S$ such that $P(x)$ is an existentially quantified propositional function on $S$. It is true if there exists $x$ in $S$ such that $P(x)$ holds true and it is false if $P(x)$ is false for all $x$ in $S$.$$\exists$$
    
    \item[] Definition: The statement ``For all'' $x$ in $S$, $P(x)$ is a universally quantified statement. It is true if $P(x)$ is true for all $x$ in $S$ and false when there exists $x$ in $S$. 
    $$\forall$$
    
    \vspace{1em}
    \newpage
    \begin{center}
        \textit{NESTED QUANTIFIERS}
    \end{center}
    
    $$\forall  \ x \ \exists \ y , \ x + y^{2} > 0$$
    
    is equal to For every x there exists a y such that $x + y^{2} > 0$ which is true \checkmark.
    
    \item[] Example: Determine whether the following is true or false.
    
    ``There is a positive integer `n' such that ''
    $$ 3n^{2} -5n +2 \text{ yields an odd \#.}$$
    
    In this case: $S = \mathbb{N}^{+} = \{1, 2,3,\ldots n\}$ $P(n) \qquad 3n^{2} -5n2 +2$ where $n$ is an odd number.
    
    Does there exist $n$ in $S$ such that $P(n)$ is true?
    
    -Try an odd and even number...
    
    \textit{even number :}

    \begin{equation}
    \boxed  {
            \begin{array}{rcl}
                     3n^{2}
                     & = \ 3 \ (4k^{2})\\
                     & = \ 2 \ (6k^{2})\\
                     \hline
                     5n
                     & = \ 5 \ (2k) \\
                     & = \ 2 \ (5k) \\
                     \hline
                    + 2 \\
            \end{array}
            }
    \end{equation}
     is even.

    \textit{odd number :}

    \begin{equation}
    \boxed  {
            \begin{array}{rcl}
                     3n^{2}\\
                     &= \ 3(2k-1)^{2}\\
                     &= \ 3(4k^{2}-4k+1)\\
                     &= \ 3(4k^{2}-4k)+3\\
                     &= \ 3(4k^{2}-4k)+4-1\\
                     &= \ 2(6k^{2} -6k +2) -1\\
            \end{array}
            }
    \end{equation}
     is odd.
     
     Thus
     this statement is false because p(n) yields an even number when given a number ($x$) that is a variation of $2k$.

\item[] Example: For all real numbers ``a'' and ``b'' and this propositional function : 
$$2ab \leq a^{2} + b^{2}$$ 
is this true or false?

$$ 0 \leq (a-b)^{2} = a^{2} - 2ab + b^{2} $$
adding $2ab$ to both sides we get 
$$2ab \leq a^{2} + b^{2} \qquad \text{which is true \checkmark.}$$

\begin{center}
In general this statement can be written as : \\
$ \forall a \ \forall b \ ( \in  \ \mathbb{R} ),  2ab \leq a^{2} + b^{2}$ \checkmark
\end{center}

\end{enumerate}
}

\end{document}

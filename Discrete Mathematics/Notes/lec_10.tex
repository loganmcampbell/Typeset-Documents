\documentclass[10pt,letterpaper]{article}
\usepackage{fullpage}
\usepackage[top=2cm, bottom=4.5cm, left=2.5cm, right=2.5cm]{geometry}
\usepackage{amsmath,amsthm,amsfonts,amssymb,amscd}
\usepackage{lastpage}
\usepackage{enumerate}
\usepackage{empheq}
\usepackage{fancyhdr}
\usepackage{mathrsfs}
\usepackage{mathtools}
\usepackage{xcolor}
\usepackage{graphicx}
\usepackage{caption}
\usepackage{listings}
\usepackage{hyperref}
\usepackage{pifont}
\newcommand{\xmark}{\ding{55}}%
\pagestyle{fancyplain}
\headheight 25pt
\lhead{LOGAN CAMPBELL \\ Discrete Mathematics : Lecture Notes (9/18/2019)}
\chead{}
\lfoot{}
\cfoot{}
\rfoot{\small\thepage}
\headsep 1.0em
\begin{document}
{
\begin{enumerate}
\item[]
    Example 1: use induction to show that $2^{n} \geq n^{2} \ \text{for} \ n = 4, 5, 6, 7, \ldots$
    
    
    step 1: 
    
    \qquad LHS $$S(4) = 2^{4} = 2^{4} = 16$$
    \qquad RHS $$S(4) = 4^{2} = 16 \qquad \checkmark \ S(4) \text{ holds...}$$
            
    step 2: Assume that S(n) holds for some generic $n\geq 4$. That is $\underbracket{2^{n} \leq n^{2}}_{S(n)}$
    multiplying both sides of S(n) by $2$ we get : 
    $$2 \cdot 2^{2} \geq n^{2} \cdot 2 $$
    Hence,
        $2^{n+1} \geq 2n^{2} = n^{2} + n^{2} $\\
        $= n^{2} \cdot n \cdot n$\\
        $= n^{2} \cdot 4n$\\
        $= n^{2} +2n + 2n \geq n^{2} + 2n + 1$
        
        \begin{center}
            OR
        \end{center}
    Show that: $2^{n+1} \geq (n+1)^{2} = n^{2} + 2n + 1$\\
    Lets add $2n+1$ on both sides of S(n) : $$2^{n}  \ \boxed{+ 2n+1} \ \geq n^{2} \ \boxed{+2n+1}.\quad \text{so, } 2^{n}(2n+1) \geq (n+1)^{2}$$
    Now, by problem 16, $2^{n} \geq 2(n+1) \text{ for } n \geq 3$.
    $$2^{1} \cdot 2^{n} = 2^{n+1} $$
    $2^{n} + 2^{n} \geq 2^{n} + (2n+1) \geq (n+1)^{2}.$
    
    
\end{enumerate}

\newpage{}
\begin{enumerate}
    \item [] Definition: Let $m$ and $n$ be integers. We say that $m$ divides $n$ if there is a integer $k$ such that: $n = mk$. In this case, we say that $n$ is divisible by $m$.
    
    \item[] Example: Show that $7^{n-1}$ is divisible by $6$. for $n=1,2,3, \ldots$
    
    step 1: $n=1 \rightarrow \quad s(1) = 7^{1} - 1$ is divisible by $6$.  
    $$7 - 1 = 6 \div 6 = 1 $$ \checkmark.
    
    step 2: Assume S(n) hold true for some generic integer n.
    $$S(n) = 7^{n} - 1$$ is divisible by 6 for some generic n. Hence there is an integer $k$ that :
    $$7^{n} - 1 = 6k$$
    Now: $7^{n+1} - 1 = 7 \underbracket{(7^{n} - 1)}_{6k} + 6$
    $$7^{n+1} - 1 = 7(6k) + 6$$
    $$\underbracket{6(7k+1)}_{\ell}$$
    Where:
    $6\ell$ and $\ell$ are integers.
    \vspace{1em}
    \hrule 
    \item[] Alternative way: $ 7^{n+1} - 1 = 7\cdot 7^{n} - 1 = (6k +1) =1$
    $$ = 7(6k + 1) -1 $$
    $$ = 7\cdot6k + 7 - 1$$
    $$ = 7\cdot6k + 6$$
    $$ = 6\underbracket{(7k +1)}_{7k+1 = \ell}$$ 
\end{enumerate}{}

}

\newpage{}
\begin{enumerate}
    \item [] Another Example: \\
    \quad Show that $3^{n} + 7^{n} - 2$ is divisible by 8, for $n = 1,2,3, \ldots$
    
    step 1: $S(1) = 3^{1} + 7^{1} - 2 = 8$, which is divisible by $8$.
    
    step 2: $S(n)$ Assume $S(n)$ holds true for some generic integer n.
    $$S(n) = 3^{n} + 7^{n} - 2 = 8k$$ where some integer is k.
    step 3: Assume $S(n+1)$ holds true for the next term.
    $$S(n+1) = 3^{1} \cdot 3^{n} + 7^{1} \cdot 7^{n} - 2 = 8 \ell$$
    for some integer $\ell$.
    
    Now: $3^{n+1} + 7^{n+1} - 2$ = $3\cdot 3^{n} + 7 \cdot 7^{n} - 2$.
    
    $$3 \cdot 3^{n} + (3+4) \cdot 7^{n} - 2$$
    $$3\cdot 3^{n} + 3 \cdot (7^{n}) + 4(7^{n}) - 2$$
    $$3 \cdot 3^{n} + 3 \cdot 7^{n} - 3 \cdot 2 + 4 \cdot 7^{n} - 2 + 6 $$
    $$3 (3^{n} + 7^{n} - 2) + 4 \cdot 7^{n} +  4$$
    $$= 3 (8k) + 4(7^{n} + 1)$$
    Now: $7^{n}$ is odd for any positive integer $n$. $\therefore 7^{n} + 1$ is even for any $n$. 
    $7^{n} + 1 = 2m$ where $m$ is an integer.
    
    So: $$3^{n+1} +7^{n+1} - 2 = 3(8k) + 4(2m)$$
    $$= 8(3k) + 8m$$
    $$=8 \underbracket{(3k+m)}_{\text{some integer } \ell}$$
    
    So: $3^{n+1} + 7^{n+1} - 2$ is divisible by 8. \checkmark.
    
    \vspace{1em}
    \hrule 
    
    \item[] Alternative way:
    
    $3^{n+1} + 7^{n+1} - 2 = \boxed{3^{n} + 7^{n} - 2 = 8k }$
    $$3^{n+1} + 7^{n+1} - 2 = 3(3^{n} - 7^{n} - 2)$$
    $$\boxed{ + 4\cdot 7^{n} + 8}$$
    $$3^{n+1} + 7^{n+1} - 2 = 3 (3^{n} + 7^{n} - 2) + 4 \cdot 7^{n} + 4$$
    $$= 3(8k) + 4(7^{n} + 1)$$
    
    
\end{enumerate}{}





\end{document}

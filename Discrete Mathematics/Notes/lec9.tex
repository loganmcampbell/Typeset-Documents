\documentclass[10pt,letterpaper]{article}
\usepackage{fullpage}
\usepackage[top=2cm, bottom=4.5cm, left=2.5cm, right=2.5cm]{geometry}
\usepackage{amsmath,amsthm,amsfonts,amssymb,amscd}
\usepackage{lastpage}
\usepackage{enumerate}
\usepackage{empheq}
\usepackage{fancyhdr}
\usepackage{mathrsfs}
\usepackage{mathtools}
\usepackage{xcolor}
\usepackage{graphicx}
\usepackage{caption}
\usepackage{listings}
\usepackage{hyperref}
\usepackage{pifont}
\newcommand{\xmark}{\ding{55}}%
\pagestyle{fancyplain}
\headheight 25pt
\lhead{LOGAN CAMPBELL \\ Discrete Mathematics : Lecture Notes (9/16/2019)}
\chead{}
\lfoot{}
\cfoot{}
\rfoot{\small\thepage}
\headsep 1.0em
\begin{document}
{
\begin{enumerate}
\item[]
    Example 1: use induction to show that $\sum \limits_{k = 1}^{n} k^{2} = \frac{n(n+1)(2n+1)}{6}$\\
    step 1 : establish S(smallest number): $$\therefore S(1) \quad //Basis \ Case$$
    $$S(1) = \sum \limits_{k=1}^{1} = 1 = 1(1+1)(2(1) +1) = 1(2)(3) = \frac{6}{6} = 1 \quad \checkmark.$$

    step 2 : Assume $S(n)$ hold for some generic (n).
    $$S(n) = 1^{2} + 2^{2} \cdots + n^{2} \rightarrow \sum^{n}_{k=1} k^{2} = \frac{n(n+1)(2n+1)}{6}$$
    \begin{center}
        \boxed{$$+(n+1)$$} You would have to add this to establish that the goal is the additional {$$+(n+1)$$}
    \end{center}
    Furthermore:
    $$ S(n)= \underbracket{\sum^{n}_{k=1} k^{2}}_{\boxed{+ (n+1)^{2}}} = \underbracket{\frac{n(n+1)(2n+1)}{6}}_{\boxed{+ (n+1)^{2}}}$$
    
    $$\sum^{n+1}_{k=1} k^{2} = \frac{n(n+1)(2n+1)}{6} + (n+1)^{2} $$
    \underline{Factor}: 
    $$ = \frac{n(n+1)(2n+1)+6(n+1)^{2}}{(6)} $$
    
    $$ = \frac{(n+1) \left[ n(2n+1) + 6(n+1) \right]}{6}$$
    
    $$ = \frac{ (n+1)\left[ 2n^{2} + 7n + 6 \right]}{6}$$
    
    $$ = \frac{ (n+1)(n+2)(2n+3) } {6} $$
    
    step 3 : establish $S(n+1)$: $$ \sum^{n+1}_{k=1} k^{2} = \ \frac{(n+1)(n+2)(2n+3)}{6}$$
\end{enumerate}

\newpage{}
\begin{enumerate}
    \item [] Example 2: Use induction to show that $$ \underbracket{ \sum \limits^{n}_{k=1}}_{n = 1\cdots \infty} k^{3} \ = \left [ \frac{n(n+1)}{2} \right] ^ {2}$$
    
    
    step 1: establish S(1) is $$ \sum \limits^{1}_{k=1} 1^{3} = 1 = 1(1+1) = \frac{2}{2} = {1}^{2} = 1 \checkmark$$
    
    step 2: establish s(n) for some generic integer n $$S(n) = \sum \limits^{n}_{k=1} k^{3} = \bigg[ \frac{(n+1)(n+1)}{2}  \bigg] ^{2}$$
    
    step 3: establish s(n+1)
    
    $$S(n+1) = \sum \limits^{n}_{k=1} k^{3} = \bigg[ \frac{(n+1)(n+1+1)}{2}  \bigg] ^{2} \rightarrow$$
    $$ (1),(2),(3), \ldots (4) \ldots + n \ldots + (n+1) =$$
    $$(1)^{3} + (2)^{3} + (3)^{3} \ldots (n)^{3} + \underbracket{+ (n+1)^{3}}_{\text{ new integer term..}} = \Bigg[ \frac{n(n+1)}{2} \Bigg]^{2} + \ldots \underbracket{+ (n+1)^{3}}_{\text{ new integer term..}}$$
    $$ \frac{n^{2}(n+1)^{2}}{4} + \frac{4(n+1)(n+1)^{2}}{4}$$
    
    thus is true \checkmark
\end{enumerate}{}

}
\end{document}

\documentclass[10pt,letterpaper]{article}
\usepackage{fullpage}
\usepackage[top=2cm, bottom=4.5cm, left=2.5cm, right=2.5cm]{geometry}
\usepackage{amsmath,amsthm,amsfonts,amssymb,amscd}
\usepackage{lastpage}
\usepackage{enumerate}
\usepackage{fancyhdr}
\usepackage{mathrsfs}
\usepackage{mathtools}
\usepackage{xcolor}
\usepackage{graphicx}
\usepackage{caption}
\usepackage{listings}
\usepackage{hyperref}
\usepackage{pifont}
\newcommand{\xmark}{\ding{55}}%
\pagestyle{fancyplain}
\headheight 25pt
\lhead{LOGAN CAMPBELL \\ Discrete Mathematics : Lecture Notes (8/28/2019)}
\chead{}
\lfoot{}
\cfoot{}
\rfoot{\small\thepage}
\headsep 1.0em
\begin{document}

{

\begin{enumerate}
    \item[]
    \begin{center}
            COMPOUNDED PROPOSITIONS
    \end{center}
    \vspace{1.0em}
    
    Example: Let $p$, $q$, and $r$ be propositions where
    
    $$p \text{ is false}$$
    $$q \text{ is true}$$
    $$r \text{ is false}$$
    
    Determine the truth value for:
    
        \quad "start inside and go out"
    \vspace{1em}
    $$ \textcolor{blue}{( p \vee \bar{r} )} \wedge \overline{ \textcolor{green}{( q \vee r )} \vee \textcolor{red}{( \overline{r \vee p })}} $$
    
    \begin{center}
        step 1 (blue) : p is false, r \{inverse\} is true : thus an `or' makes this statement \quad TRUE\\
        step 2 (green) : q is true, r is false :an `or' makes this statement \quad TRUE\\
        step 3 (red) : r is false, p is false : an `or' and an $\overline{overline}$ makes this statement : TRUE \\
        step 4 (overline) (green and red) : inverses the two conditions.
        $$\therefore$$ TRUE and $\overline{ \text{(TRUE OR TRUE)}}$ = TRUE AND FALSE $\rightarrow${ FALSE }\\
        answer = FALSE \checkmark\\
    \end{center}
    \hrulefill
    \begin{center}
    Example 
    \begin{table}[h!]
    \caption*{Give the truth table of $(\overline{p \wedge q}) \vee (r \wedge \bar{p}$) where p, q, and r are propositions...}
        \centering
    \label{}
    \begin{tabular}{c|c|c|c} % <-- Alignments: 1st column left, 2nd middle and 3rd right, with vertical lines in between
      %\textbf{Value 1} & \textbf{Value 2} & \textbf{Value 3}\\
      $p$ & $q$ & $r$ & $(\overline{p \wedge q}) \vee (r \wedge \bar{p}$)    \\
      \hline
    T & T & T & F \\
    T & T & F & F \\
    T & F & T & T \\
    T & F & F & T \\
    \hline
    F & T & T & T \\
    F & T & F & T \\
    F & F & T & T \\
    F & F & F & T \\
    
        \end{tabular}
    \end{table}
\end{center}
\vspace{1em}
\newpage{}
Define: let P and Q be compound propositions each of which are made up of propositions $P_{1} \ldots\ldots P_{n}$ then P and Q are logically equivalent if they have identical truth tables...

for each proposition $n$ can be created a table : n $\times$ $2^{n}$.
    \begin{table}[h!]
        \centering
    \label{x} % random name
    \begin{tabular}{c|c|c} % <-- Alignments: 1st column left, 2nd middle and 3rd right, with vertical lines in between
      %\textbf{Value 1} & \textbf{Value 2} & \textbf{Value 3}\\
      $P_{1} \ldots\ldots P_{n}$ & $P$ & $Q$ \\
      \hline
     $T \ldots\ldots T$ & T & T \\
    $\vdots$ $\ddots$ $\vdots$ &$\vdots$ &$\vdots$  \\
    $F \ldots \ldots F$ & F & F \\
    
        \end{tabular}
    \end{table}
$$\therefore P \equiv Q$$
\hrulefill
\vspace{1em}

Theorem: Let $p$ and $q$ be propositions.

\quad - some propositions are equivalent to other propositions such as these 2: 

\quad \#1 : $(\overline{p \vee q}) \equiv \bar{p} \wedge \bar{q}$

\quad \#2 : $(\overline{p \wedge q}) \equiv \bar{p} \vee \bar{q}$ 
\vspace{1em}

\textbf{``TEST QUESTION: IS THIS LOGICALLY EQUIVALENT?''}

*Draw the Truth-table and compare!

example:
    \begin{table}[h!]
        \centering
    \label{z} % random name
    \begin{tabular}{|c c|c|c|} % <-- Alignments: 1st column left, 2nd middle and 3rd right, with vertical lines in between
      %\textbf{Value 1} & \textbf{Value 2} & \textbf{Value 3}\\
        \hline
        p & q & $\overline{p \vee q}$ & $\bar{p} \wedge \bar{q}$\\
        \hline
        T & T & F & F \\
        T & F & F & F \\
        F & T & F & F \\
        F & F & T & T \\
        \cline{1-4}
          &  & \checkmark & \checkmark\\
         \hline
        \end{tabular}
    \end{table}

\end{enumerate}
}

\newpage
{
\begin{enumerate}
    \item []
     \begin{center}
     Conditional Propositions
     \end{center}
    
    Definition: Let $p$ and $q$ be propositions then the conditional proposition be : $p \rightarrow q$
    is the statement: ``If $p$ then $q$'' Now in $p \rightarrow q$; $p$ is the hypothesis and $q$ is the conclusion. Truth-Table below
    \begin{center} 
    \begin{table}[h!]
    \caption*{$p \rightarrow q$ }
        \centering
    \label{c}
    \begin{tabular}{|c c | c|} % <-- Alignments: 1st column left, 2nd middle and 3rd right, with vertical lines in between
      %\textbf{Value 1} & \textbf{Value 2} & \textbf{Value 3}\\
      \hline
      $p$ & $q$ &   $p \rightarrow q$  \\
      \hline
      T & T & T \\
      T & F & F \\
      F & T & T \\
      F & F & T \\
      \hline
    
        \end{tabular}
    \end{table}
\end{center}
    
    \item[] Be able to convert statements into `if - then' statements
    
    1. We will catch the bus if we run!
    
    \qquad if: if we run \quad then: then we will catch the bus.
    
    2. The cactus grows \underline{only if} it has enough sunlight.
    
    \qquad if: if the cactus grows \quad then: then it has enough sunlight.
    
    
    
\end{enumerate}
}

\end{document}

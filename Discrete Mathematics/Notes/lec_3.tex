\documentclass[10pt,letterpaper]{article}
\usepackage{fullpage}
\usepackage[top=2cm, bottom=4.5cm, left=2.5cm, right=2.5cm]{geometry}
\usepackage{amsmath,amsthm,amsfonts,amssymb,amscd}
\usepackage{lastpage}
\usepackage{enumerate}
\usepackage{fancyhdr}
\usepackage{mathrsfs}
\usepackage{mathtools}
\usepackage{xcolor}
\usepackage{graphicx}
\usepackage{caption}
\usepackage{listings}
\usepackage{hyperref}
\usepackage{pifont}
\newcommand{\xmark}{\ding{55}}%
\pagestyle{fancyplain}
\headheight 25pt
\lhead{LOGAN CAMPBELL \\ Discrete Mathematics : Lecture Notes (8/30/2019)}
\chead{}
\lfoot{}
\cfoot{}
\rfoot{\small\thepage}
\headsep 1.0em
\begin{document}

{

\begin{enumerate}
    \item[]
    \begin{center}
            COMPOUNDED PROPOSITIONS . . . [continued]
    \end{center}
    \vspace{1.0em}
    3. You may have my book when the cows come home.
    
    \qquad if: if the cows come home. \quad then: then you may have my book.
    
    4. In order to drive my car it is necessary to have fuel in it.
    
    \qquad if: if I can drive my car. \quad then: then I have fuel in it.
    \vspace{1em}
    \item[] Definition: Let $p$ and $q$ be propositions:
    
    1. The converse of $p \rightarrow q$ is \quad $q \rightarrow p$
    
    2. the contrapositive from $p \rightarrow q$ is $\bar{q} \rightarrow \bar{p}$
    \vspace{1em}
    
    an example: $\overbracket{\text{it is raining on campus}}^{p}$ | $\underbrace{\text{it is cloudy over campus.}}_{q}$
    
    $q \rightarrow p$ : if it is cloudy on campus then it is raining on campus. 
 
    $\bar{q} \rightarrow \bar{p}$ : if it is \underline{not} cloudy over campus then it is \underline{not} raining on campus.
    
    \item[] Theorem Let $p$ and $q$ be propositions.
    
    \quad 1. $(p \rightarrow q)$ $\not\equiv$ $(q \rightarrow p)$
    
    \quad 2. $(p \rightarrow q)$ $\equiv$ $\bar{q} \rightarrow \bar{p}$
    
    \begin{table}[h!]
    \centering
    \label{z} % random name
    \begin{tabular}{|cc|c|c|c|} % <-- Alignments: 1st column left, 2nd middle and 3rd right, with vertical lines in between
      %\textbf{Value 1} & \textbf{Value 2} & \textbf{Value 3}\\
        \hline
        p & q & ${p \rightarrow q}$ & $q \rightarrow p$ & $\bar{q} \rightarrow \bar{p}$ \\
        \hline
        T & T & T& T & T \\
        T & F & F& T & F \\
        F & T & T& F & T \\
        F & F & T& T & T \\
        \cline{1-5}
          %&  &  &  &  & \\
         \hline
        \end{tabular}
    \end{table}
    
    * notice they are not logically equivalent
    
    \hrulefill
    \item[] Write A Contrapositive Mathematical Expression
    
    if $\underbrace{x \not= 3}_{p}$ then $ \underbrace{x < 3 \text{ or } x > 3}_{q} $
    
    the contrapositive is:
    \quad if $x \geq 3 $ and $ x \leq 3$ then $x = 3$ \checkmark
    

\end{enumerate}
}

\end{document}
